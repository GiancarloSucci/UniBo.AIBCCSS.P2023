 \documentclass{beamer}
%
% Choose how your presentation looks.
% For more themes, color themes and font themes, see:
% http://deic.uab.es/~iblanes/beamer_gallery/index_by_theme.html
%
\mode<presentation>
{
  \usetheme{Madrid}      % or try Darmstadt, Madrid, Warsaw, ...
  \usecolortheme{seahorse} % or try albatross, beaver, crane, ...
  \usefonttheme{serif}  % or try serif, structurebold, ...
  \setbeamertemplate{navigation symbols}{}
  \setbeamertemplate{caption}[numbered]
  \setbeamertemplate{itemize/enumerate body begin}{\large}
\setbeamertemplate{itemize/enumerate subbody begin}{\large}
\setbeamertemplate{itemize/enumerate subsubbody begin}{\large}

  \usepackage{amsmath}
  \usepackage{tcolorbox}
  \usepackage[export]{adjustbox}
  \tcbuselibrary{most}
  \usepackage{arydshln}
  \usepackage{tikz}
  \usetikzlibrary{plotmarks}
  \usepackage{pgfplots}
 %\usepackage{enumitem}
%\usepackage{enumerate}
  %\usepackage[shortlabels]{enumitem}
} 


\definecolor{myblue}{RGB}{65,105,225} 
\definecolor{myorange}{RGB}{250,190,0}

\setbeamercolor{structure}{fg=white,bg=myorange}
\setbeamercolor*{palette primary}{fg=myblue,bg=myorange}
\setbeamercolor*{palette secondary}{fg=white,bg=myblue}
\setbeamercolor*{palette tertiary}{bg=myblue,fg=white}
\setbeamercolor*{palette quaternary}{fg=white,bg=myorange!50}

\setbeamercolor{frametitle}{fg=black!90!myblue}

\setbeamercolor{section in head/foot}{fg=white,bg=myblue}
\setbeamercolor{author in head/foot}{fg=black,bg=myorange}
\setbeamercolor{title in head/foot}{fg=white,bg=myblue}

\setbeamertemplate{navigation symbols}{}

\setbeamertemplate{itemize/enumerate body begin}{\large}
\setbeamertemplate{itemize/enumerate subbody begin}{\large}


\defbeamertemplate*{headline}{mytheme}
{%
  \begin{beamercolorbox}[ht=2.25ex,dp=3.75ex]{section in head/foot}
    \insertnavigation{\paperwidth}
  \end{beamercolorbox}%
}%

\defbeamertemplate*{footline}{mytheme}
{
  \leavevmode%
  \hbox{%
  \begin{beamercolorbox}[wd=.5\paperwidth,ht=2.25ex,dp=1ex,right]{author in head/foot}%
    \usebeamerfont{author in head/foot}\insertshortauthor\hspace*{2em}
  \end{beamercolorbox}%
  \begin{beamercolorbox}[wd=.5\paperwidth,ht=2.25ex,dp=1ex,left]{title in head/foot}%
    \usebeamerfont{title in head/foot}\hspace*{2em}\insertshortsubtitle\hspace*{2em}
    \insertframenumber{} / \inserttotalframenumber
  \end{beamercolorbox}}%
  \vskip0pt%
}

\usepackage[english]{babel}
%\usepackage[utf8x]{inputenc}
\usepackage{xcolor}
\usepackage{listings}
\usepackage{pgf}  
\usepackage{textpos}
\usepackage{tabulary}
\usepackage{scrextend}
\usepackage{hyperref}
\usepackage{setspace}
\usepackage{rotating}
\lstset
{
    language=[LaTeX]TeX,
    breaklines=true,
    basicstyle=\tt\scriptsize,
    %commentstyle=\color{green}
    keywordstyle=\color{blue},
    %stringstyle=\color{black}
    identifierstyle=\color{magenta},
}
\newcommand{\bftt}[1]{\textbf{\texttt{#1}}}
%\newcommand{\comment}[1]{{\color[HTML]{008080}\textit{\textbf{\texttt{#1}}}}}
\newcommand{\cmd}[1]{{\color[HTML]{008000}\bftt{#1}}}
\newcommand{\bs}{\char`\\}
\newcommand{\cmdbs}[1]{\cmd{\bs#1}}
\newcommand{\lcb}{\char '173}
\newcommand{\rcb}{\char '175}
\newcommand{\cmdbegin}[1]{\cmdbs{begin\lcb}\bftt{#1}\cmd{\rcb}}
\newcommand{\cmdend}[1]{\cmdbs{end\lcb}\bftt{#1}\cmd{\rcb}}

\newcommand{\wllogo}{\textbf{Overleaf}}

% this is where the example source files are loaded from
% do not include a trailing slash
\newcommand{\fileuri}{https://raw.githubusercontent.com/GiancarloSucci/UniBo.IDSEPC.A2022/main/A2022.IDSEPCLaTeX/}


\usepackage{stackengine}
\def\Ruble{\stackengine{.67ex}{%
  \stackengine{.48ex}{\textsf{P}}{\rule{.8ex}{.12ex}\kern.6ex}{O}{r}{F}{F}{L}%
  }{\rule{.8ex}{.12ex}\kern.6ex}{O}{r}{F}{F}{L}\kern-.1ex}



%----------------------------------------------------------------------------------------
%	TITLE PAGE
%----------------------------------------------------------------------------------------
\title[L04]{Artificial Intelligence, Blockchain, e Criptovalute nello Sviluppo Software \newline\newline
Lezione 5: Cognitive Models in Software Development -- acquisition, retention, and use of information} % The short title appears at the bottom of every slide, the full title is only on the title page

\author[{\tiny Giancarlo Succi }]{Giancarlo Succi\\\\ Dipartimento di Informatica -- Scienza e Ingegneria\\Universit\`{a} di Bologna\\
\bftt{g.succi@unibo.it}
} % Your name
\institute[unibo] % Your institution as it will appear on the bottom of every slide, may be shorthand to save space


\date{} % Date, can be changed to a custom date

\setbeamertemplate{navigation symbols}{}
\AtBeginSection[]
{
        \begin{frame}<beamer>{Outline}
                \tableofcontents[currentsection]
        \end{frame}
}
\begin{document}
\begin{frame}
\titlepage % Print the title page as the first slide

\end{frame}

%=============================================

\addtobeamertemplate{frametitle}{}{%
\begin{textblock*}{10mm}(-0.01mm,-0.95cm)
\includegraphics[width=0.9cm]{unibo-logo.png}
\end{textblock*}}

%=============================================

\begin{frame}
{\centerline{Structure of the lecture}}
\begin{itemize}
    \item The concept of cognition
    \item Early models
    \begin{itemize}
        \item Attribution theory
        \item The na\"{i}ve scientist
        \item Stereotypes
        \item Heuristics
    \end{itemize} 
    \item Impulsive / Reflective model
    \item Implications for software production
\end{itemize} 
\end{frame}

\begin{frame}
{\centerline{Cognition}}
\begin{itemize}
    \item Our goal in this lecture is to present some paradigms discussing:
    \begin{itemize}
        \item how information is acquired and stored
        \item how such information is then used to make decisions
    \end{itemize}
    \item In this context we review existing models of knowledge throughout their historical evolution
\end{itemize} 
\end{frame}


\begin{frame}
{\centerline{Using knowledge}}
\begin{itemize}
    \item We now review a series of models of know the knowledge is used to performs actions
    \item We focus on
    \begin{itemize}
        \item attribution theory and the na\"{i}ve scientist
        \item the cognitive miser 
        \item the motivated tactician
    \end{itemize}
    \item historical evolution
\end{itemize} 
\end{frame}

\begin{frame}
{\centerline{Attribution theory (1/2)}}
\begin{itemize}
    \item How do individuals attribute properties to entities they perceive?
    \item How do individuals attribute causes for what happens around them?
    \item Attribution bias/error
    \item Locus of causality
    \begin{itemize}
    \item Internal attribution
    \begin{itemize}
    \item attribution of the cause to self
    \end{itemize}
    \item External attribution
    \begin{itemize}
    \item attribution of the cause to the environment
    \end{itemize}
    \end{itemize} 
\end{itemize} 

\begin{center}
    \tiny{Taken from \url{https://en.wikipedia.org/wiki/Attribution_(psychology)}\\
    Heider, F (1944). ``Social perception and phenomenal causality.'' Psychological Review. 51 (6): 358–374.}
\end{center}
\end{frame}

\begin{frame}
{\centerline{Attribution theory (2/2)}}
\begin{itemize}
    \item Typical biases:
    \begin{itemize}
    \item internal attribution:
    \begin{itemize}
        \item positive own situations 
        \item negative situations of others
    \end{itemize} 
    \item external attribution
    \begin{itemize}
        \item negative own situations 
        \item positive situations of others
    \end{itemize} 
    \end{itemize} 
    \item Self-determination and feeling of autonomy
\end{itemize} 

\begin{center}
    \tiny{Taken from \url{https://en.wikipedia.org/wiki/Attribution_(psychology)}\\
    Heider, F (1944). ``Social perception and phenomenal causality.'' Psychological Review. 51(6):358–374.}
\end{center}
\end{frame}

\begin{frame}
{\centerline{The Na\"{i}ve Scientist}}
\begin{itemize}
    \item Humans tries to provide a simple and rational explanation of all the details of the world around them
    \item They use the information available to them that they try to compose like in a puzzle
    \item When doing so they perform attributions
\end{itemize} 
\begin{center}
    \tiny{Taken from \url{https://en.wikipedia.org/wiki/Cognitive_miser}\\
    Duane T. Wegener and Richard E. Petty (1998) ``The naive scientist revisited: Naive theories and social judgment'' Social Cognition. 16(1):1}
\end{center}

\end{frame}

\begin{frame}
{\centerline{Stereotypes}}
\begin{itemize}
    \item Humans are not always able to handle complexity
    \item Understanding complexity requires effort
    \item A stereotype is a reconstruction of the reality where complex details are simplified
    \item Stereotypes are reinforced by looking a facts that from a simplistic perspective correspond to such stereotype
    \item Stereotypes simplify the thinking process
\end{itemize} 
\begin{center}
    \tiny{Taken from \url{https://en.wikipedia.org/wiki/Cognitive_miser}}
\end{center}

\end{frame}

\begin{frame}
{\centerline{Heuristics}}
\begin{itemize}
    \item Heuristics are another approach to cope with complexity
\end{itemize} 
\begin{center}
    \tiny{Taken from \url{https://en.wikipedia.org/wiki/Cognitive_miser}\\
    Kahneman, Daniel and Tversky, Amos (1973) ``On the psychology of prediction.'' Psychological Review. 80 (4): 237–251}
\end{center}

\end{frame}



\begin{frame}
{\centerline{The Cognitive Miser}}
\begin{itemize}
    \item Humans tries to provide a simple and rational explanation of all the details of the world around them
    \item They use the information available to them that they try to compose like in a puzzle
    \item When doing so they perform attributions
\end{itemize} 
\begin{center}
    \tiny{Taken from \url{https://en.wikipedia.org/wiki/Cognitive_miser}\\
    Duane T. Wegener and Richard E. Petty (1998) ``The naive scientist revisited: Naive theories and social judgment'' Social Cognition. 16(1):1}
\end{center}

\end{frame}


\begin{frame}
{\centerline{AI?}}
\begin{itemize}
\item Modeling the human mind
\item Emulating the human mind
\begin{itemize}
\item Vision
\item Natural language processing
\item Reasoning
\item ...
\end{itemize} 
\item Computational intelligence
\begin{itemize}
\item \textit{Data science}
\item Fuzzy models
\item Machine learning
\item Granular computing
\item ...
\end{itemize}
\item Automated reasoning and formal models
\end{itemize} 
\end{frame}

\begin{frame}
{\centerline{AI? -- Our focus}}
\begin{itemize}
\item \textcolor{red}{Modeling the human mind}
\item Emulating the human mind
\begin{itemize}
\item Vision
\item Natural language processing
\item Reasoning
\item ...
\end{itemize} 
\item Computational intelligence
\begin{itemize}
\item \textcolor{red}{\textit{Data science}}
\item Fuzzy models
\item \textcolor{red}{Machine learning}
\item Granular computing
\item ...
\end{itemize}
\item Automated reasoning and formal models
\end{itemize} 
\end{frame}


\begin{frame}
{\centerline{Modeling the human mind}}
\begin{itemize}
    \item Physiological analysis
    \begin{itemize}
    \item Analysis of physiological signals
    \item Contextual analysis of situations
        \begin{itemize}
        \item Pair programming
        \item Standup meetings
        \end{itemize} 
    \end{itemize} 
    \item Psychological analysis
    \begin{itemize}
    \item Systemic thinking
    \item Distributed cognition/extended mind
    \item Dual model
    \end{itemize} 
    \item Analysis by analogy with other disciplines 
     \begin{itemize}
     \item Storytelling
    \item Painting
    \item Dancing
    \end{itemize}    
\end{itemize} 
\end{frame}


\end{document}
