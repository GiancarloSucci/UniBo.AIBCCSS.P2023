% !TeX encoding = UTF-8
% !TeX spellcheck = en_US
\documentclass[11pt, a4paper]{article}
\usepackage{graphicx} 
\usepackage[export]{adjustbox}
\usepackage[left=2cm,right=2cm,top=2cm,bottom=2cm]{geometry} 
\usepackage{array, booktabs, longtable}
\usepackage{xurl}
%\usepackage{hyperref}
\usepackage{xparse}
\usepackage{comment}
\usepackage{array}
\newcolumntype{C}[1]{>{\centering\arraybackslash}p{#1}}
\usepackage{fp}
\usepackage{ulem}
\usepackage[usenames,dvipsnames]{xcolor}



% The title of the current document to be produced.
\NewDocumentCommand{\doctitle}{}{Sillabo del Corso}
\NewDocumentCommand{\materialedidattico}{o}{%
    \IfNoValueTF{#1}
        {\redtext{\textbf{Materiale didattico~}}}
        {\redtext{\textbf{Materiale didattico:~} #1}}}
\NewDocumentCommand{\argomento}{o}{%
    \IfNoValueTF{#1}
        {\bluetext{\textbf{Argomento~}}}
        {\bluetext{\textbf{Argomento:~} #1}}}
\NewDocumentCommand{\dataeora}{o}{%
    \IfNoValueTF{#1}
        {\textcolor{ForestGreen}{\textbf{Data e ora~}}}
        {\textcolor{ForestGreen}{\textbf{Data e ora:~} #1}}}

%
\setlength{\unitlength}{1in}
\renewcommand{\arraystretch}{2}

%------------------------------------------------------------
% Import commands for both teacher and course information.  | 
% NOTE: Change your teacher and course info in these files. |
%------>------>------>------>------>------>------>------>-->|
\input{Sillabo/teacher-info}                              %|
%% ==================================
%% ===== Course-specific commands ===
%% ==================================

%- Instructions: change course info here. 
\newcommand{\semester}{Primavera 2023}

\newcommand{\csection}{Laurea Magistrale in Informatica}
\newcommand{\ponderation}{40 ore di didattica}
\newcommand{\coursetitle}{Artificial Intelligence, Blockchain e Criptovalute Nello Sviluppo Software}
\newcommand{\coursenumber}{[Course Number]}
\newcommand{\prerequisite}{Basi di matematica,  logica, programmazione e ingegneria del software}
\newcommand{\gruppoTelegram}{\url{https://t.me/+AcRzJlpSBRhmMzE0}}

                               %|   
%
%------------------------------------------------------------
%-- Import packages and custom command definitons.          |
%------>------>------>------>------>------>------>------>-->|
\input{Sillabo/packages-imports}                          %|  
\input{Sillabo/custom-commands}   
%
%---> Genereate & inject metadata                           |
\input{Sillabo/hyperef.doc.info}                          %|
%------------------------------------------------------------

\topmargin -70pt
\begin{document} 

%-------------------------------------------------------------
%-- Make the header of the document                          |
%------>------>------>------>------>------>------>------>--> |
\input{Sillabo/document-header}
  
%-------------------------------------------------------------
%-- Insert the course & teacher info                         |
%------>------>------>------>------>------>------>------>--> |
\hrule     
\vspace{.5cm}
\begin{multicols}{2}
    \begin{description}[labelindent=0.02in,leftmargin=1.25in,style=nextline]
        %--> First column:         
        \item[\textsc{Docente}:] \instructor
        \item[\textsc{Telefono}:]\phone
        \item[\textsc{E-mail}:] \email
        \item[\textsc{Ricevimento}:] \hours
        \item[\textsc{Studio}:]  {\color{darkred}\office}
        \item[\textsc{Telegram}:] \contattoTelegram
        \item[\textsc{Impegno}:] \raggedright\ponderation
        \item[\textsc{Prerequisiti}:] \prerequisite
        \item[\textsc{LM}:] \csection
        \item[\textsc{Gruppo}:] \gruppoTelegram
        %--> Second column:         
        \item[] 
    \end{description}
\end{multicols}
\hrule        
\vspace{.2cm}

 %--------  Course Description  ------------------------------
 \customsection{Scopo del corso}  
\noindent
L'insegnamento si propone di formare gli studenti, affinché alla fine del corso:
\begin{borderedsquare}
\item conoscano i principali modelli cognitivi che possano spiegare come le persone sviluppano il software,
\item siano consapevoli delle opportunità e dei limiti dell’applicazione degli strumenti dell’intelligenza artificiale per lo sviluppo del software,
\item acquisiscano familiarità a come i principi di ingegneria del software possano guidare lo sviluppo di sistemi basati sull’intelligenza artificiale,
\item padroneggino i principi del blockchain e le sue applicazioni,
\item comprendano il ruolo e le potenzialità delle criptovalute, e le problematiche ad esse associate nello sviluppo di sistemi software,
\item riescano costruire complessi modelli di produzione e di prodotto combinando i vari settori di punta degli odierni produttivi, con particolare rilevanza all’intelligenza artificiale, ai sistemi blockchain e alle criptovalute.
\end{borderedsquare}


\customsection{Descrizione del corso}  
\noindent
Negli ultimi anni si è assistito ad un sostanziale cambio di paradigma nello sviluppo del software che ha portato a ripensare processi ed ambiti applicativi. In particolare, è emerso un rinnovato interesse per l’intelligenza artificiale, con un forte interesse all’applicazione dell’apprendimento automatico nei modelli di predizione e qualità, e nell’uso di modelli cognitivi per orientare i processi di produzione. D’altro canto, è emerso un forte bisogno di identificare strumenti software adatti per gestire le piattaforme di analisi dati, che sono sempre più complesse e generano sistemi che in prima battuta sembrano di notevole portata ma poi sono difficili da far evolvere. In aggiunta a tutto questo il settore è stato permeato da una sempre maggiore distribuzione delle applicazioni e dei processi di sviluppo, intrecciato con le nuove modalità di gestione degli aspetti e delle applicazioni relative all’introduzione dei sistemi a blockchain e alle criptovalute.

\customsection{Obiettivi formativi} 
\noindent
Questo è un corso di ingegneria del software che si propone di formare gli studenti, affinché alla fine del corso:
\begin{borderedsquare}
     \setlength\itemsep{0.3em}        
\item conoscano i principali modelli cognitivi che possano spiegare come le persone sviluppano il software
\item siano consapevoli delle opportunità e dei limiti dell’applicazione degli strumenti dell’intelligenza artificiale per lo sviluppo del software
\item acquisiscano familiarità a come i principi di ingegneria del software possano guidare lo sviluppo di sistemi basati sull’intelligenza artificiale
\item padroneggino i principi del blockchain e le sue applicazioni
\item comprendano il ruolo e le potenzialità delle criptovalute, e le problematiche ad esse associate nello sviluppo di sistemi software
\item riescano costruire complessi modelli di produzione e di prodotto combinando i vari settori di punta degli odierni produttivi, con particolare rilevanza all’intelligenza artificiale, ai sistemi blockchain e alle criptovalute
\end{borderedsquare}
        
%\clearpage    

\customsection{Prerequisiti}
Pur non avendo alcun prerequisito formale, il corso si caratterizza come corso di ingegneria del software e quindi \textbf{non} presenterà gli aspetti fondanti dell’intelligenza artificiale e dell’apprendimento automatico; all’uopo si raccomanda agli studenti che non abbiano tali competenze di seguire preventivamente il corso di Deep Learning, cod. 91250, del prof. Asperti. Inoltre, essendo un corso di Laurea Magistrale in informatica ci si aspetta che gli studenti abbiano le opportune basi di  matematica, logica, statistica, programmazione e ingegneria del software.


\customsection{Valutazione}    
\begin{itemize}
\item Nella prima sessione di esame, lo studente in corso può scegliere tra un orale onnicomprensivo e un progetto (eventualmente da svolgersi in gruppo) sui temi del corso ed assegnato dal docente; ci saranno inoltre test in classe per valutare l'apprendimento incrementale del materiale presentato.
\item nelle successive sessioni, la valutazione si baserà su un orale onnicomprensivo.
\end{itemize}

\customsection{Possibili progetto}    
\noindent
Simulare il comportamento di trader nel mercato delle criptovalute utilizzando modelli regressivi con funzione di costo basata sulla distanza di Kendall.


\customsection{Testi}  
%---------------------------------
%--> List of recommended textbooks. 
\begin{itemize}[itemsep=4pt,parsep=0pt,topsep=1pt,partopsep=1pt]
	\item[\color{darkblue}\faNewspaper] \textbf{\textsc{Risorse disponibili online:}}
	Materiale presentato in classe, altri riferimenti comunicati dal docente durante le lezioni.  
		
	\item[\color{darkblue}\faBook] \textbf{\textsc{Libro di testo:}} Non c'\`{e} un libro di testo obbligatorio. % Qui nel seguito si presenta una serie di lettura consigliate.
\end{itemize}
\begin{comment}
\vspace{.4cm}
\hrule
\vspace{.4cm}
\begin{minipage}[b]{0.17\linewidth}          
	\includegraphics[width=.95\linewidth]{Sillabo/images/TextAnalysisInPythonForSocialScientists.jpg}
\end{minipage}\hfill
\begin{minipage}[b]{0.75\linewidth}          
	\noindent \textbf{Titolo:} Text Analysis in Python for Social Scientists - Discovery and Exploration \\
	\textbf{Autore:} Dirk Hovy \\
	\textbf{Casa editrice:} Cambridge University Press, publication year: 2021  \\
	\textbf{ISBN Online:} 9781108873352\\   
	\vspace{1cm}
	~
\end{minipage}
\vspace{.18cm}
\hrule
\vspace{.4cm}
\begin{minipage}[b]{0.17\linewidth}          
	\includegraphics[width=.95\linewidth]{Sillabo/images/AppliedComputationalThinkingWithPython.jpeg}
\end{minipage}\hfill
\begin{minipage}[b]{0.75\linewidth}          
	\noindent \textbf{Titolo:} Applied Computational Thinking with Python \\
	\textbf{Autore:} Sof\'{i}a De Jes\'{u}s , Dayrene Martinez \\
	\textbf{Casa editrice:} Packt Publishing; publication year: 2020  \\
	\textbf{ISBN-13:} 978-1839219436\\  
	\vspace{1cm}
	~
\end{minipage}
\vspace{.18cm}
\hrule
\vspace{.4cm}
\begin{minipage}[t]{0.17\linewidth}          
	\includegraphics[trim={0 0 0 1cm},clip, width=.95\linewidth,]{Sillabo/images/1200px-Wikibooks-logo.png}
\end{minipage}\hfill
\begin{minipage}[b]{0.75\linewidth}  
    \vspace{.5cm}
	\noindent \textbf{Titolo:} The LaTeX Wikibook  \\
	\textbf{Autore:} Multipli \\
	\textbf{Casa editrice:} Wikibooks community  \\
	\textbf{Disponibile online:} \url{https://en.wikibooks.org/wiki/LaTeX}\\
	\vspace{1cm}
	~
\end{minipage}
\vspace{.18cm}
\hrule

\\
\end{comment}

%--------  Required Software and Material ------------------------------
\vspace{.4cm}

\customsection{Strumenti software utilizzati }  
\begin{itemize}[itemsep=2pt,parsep=0pt,topsep=2pt,partopsep=2pt]
	%    \item[\color{darkblue}\faCoffee] Java 7 or 8 (32 or 64 bits)
	\item[\color{darkblue}\faLaptopCode] \textbf{Sistemi operativi:} \faWindows {} Windows  10,  \faLinux {} Linux, \textcolor{vanierred}{\textbf{o}} \faApple {} macOS 
	\item[\color{darkblue}\faCode] \textbf{IDE per Python:} \faUnity a scelta dello studente, da coordinare con il parallelo corso di laboratorio di programmazione
	\item [{\color{darkblue}\faChrome}] \textbf{Web Browser:} Chrome, Safari o Firefox.   
	\item[{\color{darkblue} \faWpforms}] LaTex per la produzione e l'analisi di documenti; all'uopo gli studenti sono incoraggiati a crearsi un account su \url{overleaf.com} e comunicarlo al docente
\end{itemize}   

\customsection{Regole di comportamento} 
\begin{itemize}[itemsep=2.5pt,parsep=0pt,topsep=8pt,partopsep=4pt]
	\item[{ \color{darkblue} \faLaptop \faMobile \faHeadphones}] In classe, l'uso di cellulari, di computer e di cuffie, quando non richiesto dal docente, \`{e}  \underline{vietato}. I cellulari vanno spenti e riposti in luogo sicuro non nei banchi.
	
	\item[{\color{darkblue} \faEdit}] Il docente si aspetta che gli studenti si presentino con carta e penna/matita e prendano attivamente appunti.
	\item[{\color{darkblue} \faRocketchat}] Gli studenti sono ammessi alle lezioni solo all'inizio e durante le lezioni devono astenersi dal parlare tra loro, se non quando esplicitamente richiesto dal docente.
	
	\item[{\color{darkred} \faThumbsDown}] Il plagiarismo negli elaborati comporter\`{a} la bocciatura nell'esame e le ulteriori penalit\`{a} previste nei competenti regolamenti accademici.
\end{itemize}

%\customsection{\sout{Aula per le lezioni}}

\customsection{Luogo di svolgimento delle lezioni} 
\noindent
AULA E3 \\
Piano Terra  \\
Edificio in Bo - via Q. Filopanti 1-3 \\
Viale Quirico Filopanti, 1-3 - Bologna \\
\href{https://goo.gl/maps/zXt9hYgFqeWdwXZYA}{Posizione dell'ingresso su Google Maps}\\


\customsection{Programma di massima delle lezioni}
\newcounter{lezione}
\setcounter{lezione}{0}
\addtocounter{lezione}{1}

%TODO: split implementation into 3 builds.
\renewcommand{\arraystretch}{1.5} % this reduces the vertical spacing between rows    
\noindent\begin{longtable}{|C{1cm}|p{15cm}|}
\toprule
	\thead{\color{darkblue} Lezione} & \thead{\argomento, \materialedidattico \& \dataeora} \\ 
	%---s Load the table body: dynamic table content. 
\midrule
    \endhead
	\centering\textbf{\thelezione \addtocounter{lezione}{1}} &   \argomento[Presentazione del corso, introduzione all'ingegneria del software - problemi tame e wicked]  \\
	& \materialedidattico[\href{https://github.com/GiancarloSucci/UniBo.AIBCCSS.P2023/blob/main/P2023.AIBCCSS.L01.L02.L03.IlConcettoDiSoftware.pdf}{Lucidi presentati dal docente sull'introduzione all'ingegneria del software - problemi tame e wicked}] \\
	& \dataeora[1 marzo 2023] \\
	\hline    
	\centering\textbf{\thelezione \addtocounter{lezione}{1}} &   \argomento[Processi di sviluppo software]  \\
	& \materialedidattico \\
	& \dataeora[2 marzo 2023] \\
	\hline    
	\centering\textbf{\thelezione \addtocounter{lezione}{1}} &   \argomento[Processi di sviluppo software e modelli agili]  \\
	& \materialedidattico \\
	& \dataeora[8 marzo 2023] \\
	\hline    
	\centering\textbf{\thelezione \addtocounter{lezione}{1}} &   \argomento[Modelli cognitivi -- misurazione dell'attivit\`{a} cerebrale]  \\
	& \materialedidattico \\
	& \dataeora[9 marzo 2023] \\
	\hline    
	\textbf{\thelezione \addtocounter{lezione}{1}} &   \argomento[Modelli cognitivi -- impulsive / reflective]  \\
	& \materialedidattico \\
	& \dataeora[15 marzo 2023] \\
	\hline    
	\textbf{\thelezione \addtocounter{lezione}{1}} &   \argomento[Modelli cognitivi -- intelligenza distribuita]  \\
	& \materialedidattico \\
	& \dataeora[16 marzo 2023] \\
	\hline    
	\textbf{\thelezione \addtocounter{lezione}{1}} &   \argomento[Modelli cognitivi -- il disegno come strumento di elicitazione della conoscenza]  \\
	& \materialedidattico \\
	& \dataeora[22 marzo 2023] \\
	\hline    
	\textbf{\thelezione \addtocounter{lezione}{1}} &   \argomento[Modelli cognitivi -- arte e software]  \\
	& \materialedidattico \\
	& \dataeora[23 marzo 2023] \\
	\hline    
	\textbf{\thelezione \addtocounter{lezione}{1}} &   \argomento[Statistica descrittiva, statistica inferenziale, intelligenza artificiale]  \\
	& \materialedidattico \\
	& \dataeora[29 marzo 2023] \\
	\hline    
	\textbf{\thelezione \addtocounter{lezione}{1}} &   \argomento[Regressione lineare e correlazione]  \\
	& \materialedidattico \\
	& \dataeora[30 marzo 2023] \\
	\hline    
	\textbf{\thelezione \addtocounter{lezione}{1}} &   \argomento[Regressione logistica]  \\
	& \materialedidattico  \\
	& \dataeora[5 aprile 2023] \\
	\hline
	\textbf{\thelezione \addtocounter{lezione}{1}} &   \argomento[Il test di Kolmogorov Smirnov]  \\
& \materialedidattico \\
	& \dataeora[12 aprile 2023] \\
	\hline
	\textbf{\thelezione \addtocounter{lezione}{1}} &   \argomento[Meta-analisi]  \\
& \materialedidattico  \\
	& \dataeora[13 aprile 2023] \\
	\hline
	\textbf{\thelezione \addtocounter{lezione}{1}} &   \argomento[Particle swarm optimization e riduzione di dimensionalit\`{a}]  \\

   & \materialedidattico  \\
	& \dataeora[19 aprile 2023] \\
	\hline    
   \textbf{\thelezione \addtocounter{lezione}{1}} &   \argomento[Regressione con distanza di Kendall]  \\
	& \materialedidattico \\
	& \dataeora[20 aprile 2023] \\
	\hline    
	\textbf{\thelezione \addtocounter{lezione}{1}} &   \argomento[Blockchain, struttura e criptovalute]  \\
	& \materialedidattico \\
	& \dataeora[3 maggio 2023] \\
	\hline    
	\textbf{\thelezione \addtocounter{lezione}{1}} &   \argomento[Criptovalute e sviluppo software]  \\
	& \materialedidattico \\
	& \dataeora[4 maggio 2023] \\
	\hline    
	\textbf{\thelezione \addtocounter{lezione}{1}} &   \argomento[Piattaforma per le criptovalute]  \\
	& \materialedidattico \\
	& \dataeora[10 maggio 2023] \\
	\hline    
	\textbf{\thelezione \addtocounter{lezione}{1}} &   \argomento[Presentazione dei progetti]  \\
	& \materialedidattico \\
	& \dataeora[11 maggio 2023] \\
	\hline    
	\textbf{\thelezione \addtocounter{lezione}{1}} &   \argomento[Presentazione dei progetti]  \\
	& \materialedidattico \\
	& \dataeora[17 maggio 2023] \\
	\hline    

\end{longtable}   

\vspace{1cm}
\noindent \textcolor{red}{\textbf{\underline{NOTA:}}} Il programma sopra specificato e l'aula delle lezioni possono essere modificati per esigenze organizzative e didattiche. 

\end{document} 